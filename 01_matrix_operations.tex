% Options for packages loaded elsewhere
\PassOptionsToPackage{unicode}{hyperref}
\PassOptionsToPackage{hyphens}{url}
\PassOptionsToPackage{dvipsnames,svgnames,x11names}{xcolor}
%
\documentclass[
]{krantz}

\usepackage{amsmath,amssymb}
\usepackage{lmodern}
\usepackage{iftex}
\ifPDFTeX
  \usepackage[T1]{fontenc}
  \usepackage[utf8]{inputenc}
  \usepackage{textcomp} % provide euro and other symbols
\else % if luatex or xetex
  \usepackage{unicode-math}
  \defaultfontfeatures{Scale=MatchLowercase}
  \defaultfontfeatures[\rmfamily]{Ligatures=TeX,Scale=1}
\fi
% Use upquote if available, for straight quotes in verbatim environments
\IfFileExists{upquote.sty}{\usepackage{upquote}}{}
\IfFileExists{microtype.sty}{% use microtype if available
  \usepackage[]{microtype}
  \UseMicrotypeSet[protrusion]{basicmath} % disable protrusion for tt fonts
}{}
\makeatletter
\@ifundefined{KOMAClassName}{% if non-KOMA class
  \IfFileExists{parskip.sty}{%
    \usepackage{parskip}
  }{% else
    \setlength{\parindent}{0pt}
    \setlength{\parskip}{6pt plus 2pt minus 1pt}}
}{% if KOMA class
  \KOMAoptions{parskip=half}}
\makeatother
\usepackage{xcolor}
\setlength{\emergencystretch}{3em} % prevent overfull lines
\setcounter{secnumdepth}{-\maxdimen} % remove section numbering
% Make \paragraph and \subparagraph free-standing
\ifx\paragraph\undefined\else
  \let\oldparagraph\paragraph
  \renewcommand{\paragraph}[1]{\oldparagraph{#1}\mbox{}}
\fi
\ifx\subparagraph\undefined\else
  \let\oldsubparagraph\subparagraph
  \renewcommand{\subparagraph}[1]{\oldsubparagraph{#1}\mbox{}}
\fi

\usepackage{color}
\usepackage{fancyvrb}
\newcommand{\VerbBar}{|}
\newcommand{\VERB}{\Verb[commandchars=\\\{\}]}
\DefineVerbatimEnvironment{Highlighting}{Verbatim}{commandchars=\\\{\}}
% Add ',fontsize=\small' for more characters per line
\usepackage{framed}
\definecolor{shadecolor}{RGB}{241,243,245}
\newenvironment{Shaded}{\begin{snugshade}}{\end{snugshade}}
\newcommand{\AlertTok}[1]{\textcolor[rgb]{0.68,0.00,0.00}{#1}}
\newcommand{\AnnotationTok}[1]{\textcolor[rgb]{0.37,0.37,0.37}{#1}}
\newcommand{\AttributeTok}[1]{\textcolor[rgb]{0.40,0.45,0.13}{#1}}
\newcommand{\BaseNTok}[1]{\textcolor[rgb]{0.68,0.00,0.00}{#1}}
\newcommand{\BuiltInTok}[1]{\textcolor[rgb]{0.00,0.23,0.31}{#1}}
\newcommand{\CharTok}[1]{\textcolor[rgb]{0.13,0.47,0.30}{#1}}
\newcommand{\CommentTok}[1]{\textcolor[rgb]{0.37,0.37,0.37}{#1}}
\newcommand{\CommentVarTok}[1]{\textcolor[rgb]{0.37,0.37,0.37}{\textit{#1}}}
\newcommand{\ConstantTok}[1]{\textcolor[rgb]{0.56,0.35,0.01}{#1}}
\newcommand{\ControlFlowTok}[1]{\textcolor[rgb]{0.00,0.23,0.31}{#1}}
\newcommand{\DataTypeTok}[1]{\textcolor[rgb]{0.68,0.00,0.00}{#1}}
\newcommand{\DecValTok}[1]{\textcolor[rgb]{0.68,0.00,0.00}{#1}}
\newcommand{\DocumentationTok}[1]{\textcolor[rgb]{0.37,0.37,0.37}{\textit{#1}}}
\newcommand{\ErrorTok}[1]{\textcolor[rgb]{0.68,0.00,0.00}{#1}}
\newcommand{\ExtensionTok}[1]{\textcolor[rgb]{0.00,0.23,0.31}{#1}}
\newcommand{\FloatTok}[1]{\textcolor[rgb]{0.68,0.00,0.00}{#1}}
\newcommand{\FunctionTok}[1]{\textcolor[rgb]{0.28,0.35,0.67}{#1}}
\newcommand{\ImportTok}[1]{\textcolor[rgb]{0.00,0.46,0.62}{#1}}
\newcommand{\InformationTok}[1]{\textcolor[rgb]{0.37,0.37,0.37}{#1}}
\newcommand{\KeywordTok}[1]{\textcolor[rgb]{0.00,0.23,0.31}{#1}}
\newcommand{\NormalTok}[1]{\textcolor[rgb]{0.00,0.23,0.31}{#1}}
\newcommand{\OperatorTok}[1]{\textcolor[rgb]{0.37,0.37,0.37}{#1}}
\newcommand{\OtherTok}[1]{\textcolor[rgb]{0.00,0.23,0.31}{#1}}
\newcommand{\PreprocessorTok}[1]{\textcolor[rgb]{0.68,0.00,0.00}{#1}}
\newcommand{\RegionMarkerTok}[1]{\textcolor[rgb]{0.00,0.23,0.31}{#1}}
\newcommand{\SpecialCharTok}[1]{\textcolor[rgb]{0.37,0.37,0.37}{#1}}
\newcommand{\SpecialStringTok}[1]{\textcolor[rgb]{0.13,0.47,0.30}{#1}}
\newcommand{\StringTok}[1]{\textcolor[rgb]{0.13,0.47,0.30}{#1}}
\newcommand{\VariableTok}[1]{\textcolor[rgb]{0.07,0.07,0.07}{#1}}
\newcommand{\VerbatimStringTok}[1]{\textcolor[rgb]{0.13,0.47,0.30}{#1}}
\newcommand{\WarningTok}[1]{\textcolor[rgb]{0.37,0.37,0.37}{\textit{#1}}}

\providecommand{\tightlist}{%
  \setlength{\itemsep}{0pt}\setlength{\parskip}{0pt}}\usepackage{longtable,booktabs,array}
\usepackage{calc} % for calculating minipage widths
% Correct order of tables after \paragraph or \subparagraph
\usepackage{etoolbox}
\makeatletter
\patchcmd\longtable{\par}{\if@noskipsec\mbox{}\fi\par}{}{}
\makeatother
% Allow footnotes in longtable head/foot
\IfFileExists{footnotehyper.sty}{\usepackage{footnotehyper}}{\usepackage{footnote}}
\makesavenoteenv{longtable}
\usepackage{graphicx}
\makeatletter
\def\maxwidth{\ifdim\Gin@nat@width>\linewidth\linewidth\else\Gin@nat@width\fi}
\def\maxheight{\ifdim\Gin@nat@height>\textheight\textheight\else\Gin@nat@height\fi}
\makeatother
% Scale images if necessary, so that they will not overflow the page
% margins by default, and it is still possible to overwrite the defaults
% using explicit options in \includegraphics[width, height, ...]{}
\setkeys{Gin}{width=\maxwidth,height=\maxheight,keepaspectratio}
% Set default figure placement to htbp
\makeatletter
\def\fps@figure{htbp}
\makeatother

\makeatletter
\makeatother
\makeatletter
\makeatother
\makeatletter
\@ifpackageloaded{caption}{}{\usepackage{caption}}
\AtBeginDocument{%
\ifdefined\contentsname
  \renewcommand*\contentsname{Table of contents}
\else
  \newcommand\contentsname{Table of contents}
\fi
\ifdefined\listfigurename
  \renewcommand*\listfigurename{List of Figures}
\else
  \newcommand\listfigurename{List of Figures}
\fi
\ifdefined\listtablename
  \renewcommand*\listtablename{List of Tables}
\else
  \newcommand\listtablename{List of Tables}
\fi
\ifdefined\figurename
  \renewcommand*\figurename{Figure}
\else
  \newcommand\figurename{Figure}
\fi
\ifdefined\tablename
  \renewcommand*\tablename{Table}
\else
  \newcommand\tablename{Table}
\fi
}
\@ifpackageloaded{float}{}{\usepackage{float}}
\floatstyle{ruled}
\@ifundefined{c@chapter}{\newfloat{codelisting}{h}{lop}}{\newfloat{codelisting}{h}{lop}[chapter]}
\floatname{codelisting}{Listing}
\newcommand*\listoflistings{\listof{codelisting}{List of Listings}}
\makeatother
\makeatletter
\@ifpackageloaded{caption}{}{\usepackage{caption}}
\@ifpackageloaded{subcaption}{}{\usepackage{subcaption}}
\makeatother
\makeatletter
\@ifpackageloaded{tcolorbox}{}{\usepackage[many]{tcolorbox}}
\makeatother
\makeatletter
\@ifundefined{shadecolor}{\definecolor{shadecolor}{rgb}{.97, .97, .97}}
\makeatother
\makeatletter
\makeatother
\ifLuaTeX
  \usepackage{selnolig}  % disable illegal ligatures
\fi
\IfFileExists{bookmark.sty}{\usepackage{bookmark}}{\usepackage{hyperref}}
\IfFileExists{xurl.sty}{\usepackage{xurl}}{} % add URL line breaks if available
\urlstyle{same} % disable monospaced font for URLs
\hypersetup{
  colorlinks=true,
  linkcolor={blue},
  filecolor={Maroon},
  citecolor={Blue},
  urlcolor={Blue},
  pdfcreator={LaTeX via pandoc}}

\author{}
\date{}

\begin{document}
\ifdefined\Shaded\renewenvironment{Shaded}{\begin{tcolorbox}[breakable, frame hidden, borderline west={3pt}{0pt}{shadecolor}, enhanced, boxrule=0pt, interior hidden, sharp corners]}{\end{tcolorbox}}\fi

\hypertarget{matrix-operations}{%
\section{Matrix Operations}\label{matrix-operations}}

Addition, subtraction, and multiplication. These are all things you
already know how to do with scalars. What happens, though, if you want
to multiply two different matrices together. Does that simple, scalar
operation still translate if you have a \(2x3\) matrix and a \(3x2\)
matrix? If you said, ``Yes!'', or if words like matrix and scalar make
you break out in a sweat, then this chapter is for you! Maybe you've
encountered these concepts before, possibly in the first 3 weeks of a
graduate statistics course; you left that class confused, angry, and
wondering why you would be subjected to such nonsense. If that sounds
familiar, this chapter is for you. If you found Linear Algebra easy,
then you can comfortably skip this chapter and see if Chapter 2 might be
where you want to start.

Matrix operations, especially multiplication, are critical for
understanding what comes throughout the rest of the book. Knowing the
underlying mechanics of matrix operations helps to demystify several
issues that you might run into with your models. You'll frequently see
model output tell you how many records were deleted due to missingness.
You'll then find yourself wondering why is missingness such a problem
that entire rows get deleted from your model? As we progress through the
next several chapters, those issues and more will become much clearer.

Before we get into any operations, though, let's make sure we are
together on some concepts.

A \emph{scalar} is a single value. It might help if you think about a
scalar as a single block @ref(fig:numbered\_block)

\begin{Shaded}
\begin{Highlighting}[]
\NormalTok{scalar\_example }\OtherTok{\textless{}{-}} \DecValTok{1}
\end{Highlighting}
\end{Shaded}

\begin{Shaded}
\begin{Highlighting}[]
\NormalTok{scalar\_example }\OperatorTok{=} \DecValTok{1}
\end{Highlighting}
\end{Shaded}

Just like we can line blocks up on the floor, we can put our scalars
together to form a \emph{vector} @ref(fig:vector\_blocks). A vector is a
collection of scalars with a length of \textbf{n}.

\begin{Shaded}
\begin{Highlighting}[]
\NormalTok{vector\_example }\OtherTok{\textless{}{-}} \DecValTok{1}\SpecialCharTok{:}\DecValTok{5}
\end{Highlighting}
\end{Shaded}

This vector containing values from 1 to 5 would have a length of 5.

\begin{Shaded}
\begin{Highlighting}[]
\NormalTok{vector\_example }\OperatorTok{=} \BuiltInTok{range}\NormalTok{(}\DecValTok{0}\NormalTok{, }\DecValTok{5}\NormalTok{)}
\end{Highlighting}
\end{Shaded}

Now, we can take a few of our block vectors and assemble them into a
\emph{matrix}. A matrix is a 2 dimensional collection of vectors.

@ref(fig:matrix\_blocks)

\[
\begin{bmatrix}
1 & 2 & 3\\
4 & 5 & 6
\end{bmatrix}
\]

If you think about most tables you've ever seen, you'll see that the
simple matrix looks remarkably familiar!

A matrix has 2 dimensions, rows and columns. When we talk about the
dimensions of a matrix, we always make note of the rows first, followed
by the columns. This matrix has 2 rows and 3 columns; therefore, we have
a \(2x3\) matrix.

\hypertarget{addition}{%
\subsection{Addition}\label{addition}}

Matrix addition, along with subtraction, is the easiest concept when
dealing with matrices. While it is easy to grasp, you will not find it
featured as prominently as matrix multiplication.

There is one rule for matrix addition: the matrices need to have the
same dimensions.

Let's check out these two matrices:

\[
\stackrel{\mbox{Matrix A}}{
\begin{bmatrix}
1_{11} & 2_{12} & 3_{13}\\
4_{21} & 5_{22} & 6_{23}
\end{bmatrix}
}  
\ 
\stackrel{\mbox{Matrix B}}{
\begin{bmatrix}
7_{11} & 8_{12} & 9_{13}\\
9_{21} & 8_{22} & 7_{23}
\end{bmatrix} 
}
\]

You probably noticed that we gave each scalar within the matrix a label
associated with its row and column position. We can use these to see how
we will produce the new matrix:

Now, we can set this up as an addition problem to produce Matrix C:

\[
\stackrel{\mbox{Matrix A}}{
\begin{bmatrix}
1_{11} & 2_{12} & 3_{13}\\
4_{21} & 5_{22} & 6_{23}
\end{bmatrix}
}  
+ 
\stackrel{\mbox{Matrix B}}{
\begin{bmatrix}
7_{11} & 8_{12} & 9_{13}\\
9_{21} & 8_{22} & 7_{23}
\end{bmatrix} 
}
=
\stackrel{\mbox{Matrix C}}{
\begin{bmatrix}
A_{11} + B_{11}& A_{12} + B_{12} & A_{13} + B_{13}\\
A_{21} + B_{21}& A_{22} + B_{22} & A_{23} + B_{23}
\end{bmatrix}
}
\]

Now we can pull in the real numbers:

\[
\stackrel{\mbox{Matrix A}}{
\begin{bmatrix}
1_{11} & 2_{12} & 3_{13}\\
4_{21} & 5_{22} & 6_{23}
\end{bmatrix}
}  
+ 
\stackrel{\mbox{Matrix B}}{
\begin{bmatrix}
7_{11} & 8_{12} & 9_{13}\\
9_{21} & 8_{22} & 7_{23}
\end{bmatrix} 
}
=
\stackrel{\mbox{Matrix C}}{
\begin{bmatrix}
1 + 7  & 2 + 8 & 3 + 9\\
4 + 9 & 5 + 8 & 6 + 7
\end{bmatrix}
}
\]

Giving us Matrix C:

\[
\stackrel{\mbox{Matrix A}}{
\begin{bmatrix}
1_{11} & 2_{12} & 3_{13}\\
4_{21} & 5_{22} & 6_{23}
\end{bmatrix}
}  
+ 
\stackrel{\mbox{Matrix B}}{
\begin{bmatrix}
7_{11} & 8_{12} & 9_{13}\\
9_{21} & 8_{22} & 7_{23}
\end{bmatrix} 
}
=
\stackrel{\mbox{Matrix C}}{
\begin{bmatrix}
8 & 10 & 12 \\
13 & 13 & 13
\end{bmatrix}
}
\]

\hypertarget{subtraction}{%
\subsection{Subtraction}\label{subtraction}}

Take everything that you just saw with addition and replace it with
subtraction!

Just like addition, every matrix needs to have the same dimensions if
you are going to use subtraction.

Let's see those two matrices again and cast it as subtraction problem:

\[
\stackrel{\mbox{Matrix A}}{
\begin{bmatrix}
1_{11} & 2_{12} & 3_{13}\\
4_{21} & 5_{22} & 6_{23}
\end{bmatrix}
}
-
\stackrel{\mbox{Matrix B}}{
\begin{bmatrix}
7_{11} & 8_{12} & 9_{13}\\
9_{21} & 8_{22} & 7_{23}
\end{bmatrix} 
}
=
\stackrel{\mbox{Matrix C}}{
\begin{bmatrix}
A_{11} - B_{11}& A_{12} - B_{12} & A_{13} - B_{13}\\
A_{21} - B_{21}& A_{22} - B_{22} & A_{23} - B_{23}
\end{bmatrix}
}
\]

And now we can substitute in the real numbers:

\[
\stackrel{\mbox{Matrix A}}{
\begin{bmatrix}
1_{11} & 2_{12} & 3_{13}\\
4_{21} & 5_{22} & 6_{23}
\end{bmatrix}
}
-
\stackrel{\mbox{Matrix B}}{
\begin{bmatrix}
7_{11} & 8_{12} & 9_{13}\\
9_{21} & 8_{22} & 7_{23}
\end{bmatrix} 
}
=
\stackrel{\mbox{Matrix C}}{
\begin{bmatrix}
1 - 7 & 2 - 8 & 3 - 9\\
4 - 9 & 5 - 8 & 6 - 7
\end{bmatrix}
}
\]

And end with this matrix:

\[
\stackrel{\mbox{Matrix A}}{
\begin{bmatrix}
1_{11} & 2_{12} & 3_{13}\\
4_{21} & 5_{22} & 6_{23}
\end{bmatrix}
}
-
\stackrel{\mbox{Matrix B}}{
\begin{bmatrix}
7_{11} & 8_{12} & 9_{13}\\
9_{21} & 8_{22} & 7_{23}
\end{bmatrix} 
}
=
\stackrel{\mbox{Matrix C}}{
\begin{bmatrix}
-6 & -6 & -6 \\
-5 & -3 & -1
\end{bmatrix}
}
\]

Adding and subtracting matrices in R and Python is pretty simple.

In R, we can create a matrix a few ways: with the matrix function or by
row binding numeric vectors.

\begin{Shaded}
\begin{Highlighting}[]
\NormalTok{matrix\_A }\OtherTok{\textless{}{-}} \FunctionTok{rbind}\NormalTok{(}\DecValTok{1}\SpecialCharTok{:}\DecValTok{3}\NormalTok{, }
                  \DecValTok{4}\SpecialCharTok{:}\DecValTok{6}\NormalTok{)}

\CommentTok{\# The following is an equivalent}
\CommentTok{\# to rbind:}
\CommentTok{\# matrix\_A \textless{}{-} matrix(c(1:3, 4:6), }
\CommentTok{\#                    nrow = 2, }
\CommentTok{\#                    ncol = 3, byrow = TRUE)}

\NormalTok{matrix\_B }\OtherTok{\textless{}{-}} \FunctionTok{rbind}\NormalTok{(}\DecValTok{7}\SpecialCharTok{:}\DecValTok{9}\NormalTok{, }
                  \DecValTok{9}\SpecialCharTok{:}\DecValTok{7}\NormalTok{)}
\end{Highlighting}
\end{Shaded}

Once we have those matrices created, we can use the standard \texttt{+}
and \texttt{-} signs to add and subtract:

\begin{Shaded}
\begin{Highlighting}[]
\NormalTok{matrix\_A }\SpecialCharTok{+}\NormalTok{ matrix\_B}
\end{Highlighting}
\end{Shaded}

\begin{verbatim}
     [,1] [,2] [,3]
[1,]    8   10   12
[2,]   13   13   13
\end{verbatim}

\begin{Shaded}
\begin{Highlighting}[]
\NormalTok{matrix\_A }\SpecialCharTok{{-}}\NormalTok{ matrix\_B}
\end{Highlighting}
\end{Shaded}

\begin{verbatim}
     [,1] [,2] [,3]
[1,]   -6   -6   -6
[2,]   -5   -3   -1
\end{verbatim}

The task is just as easy in Python. We will import \texttt{numpy} and
then use the \texttt{matrix} method to create the matrices:

\begin{Shaded}
\begin{Highlighting}[]
\ImportTok{import}\NormalTok{ numpy }\ImportTok{as}\NormalTok{ np}

\NormalTok{matrix\_A }\OperatorTok{=}\NormalTok{ np.matrix(}\StringTok{\textquotesingle{}1 2 3; 4 5 6\textquotesingle{}}\NormalTok{)}

\NormalTok{matrix\_B }\OperatorTok{=}\NormalTok{ np.matrix(}\StringTok{\textquotesingle{}7 8 9; 9 8 7\textquotesingle{}}\NormalTok{)}
\end{Highlighting}
\end{Shaded}

Just like R, we can use \texttt{+} and \texttt{-} on those matrices.

\begin{Shaded}
\begin{Highlighting}[]
\NormalTok{matrix\_A }\OperatorTok{+}\NormalTok{ matrix\_B}
\end{Highlighting}
\end{Shaded}

\begin{verbatim}
matrix([[ 8, 10, 12],
        [13, 13, 13]])
\end{verbatim}

\begin{Shaded}
\begin{Highlighting}[]
\NormalTok{matrix\_A }\OperatorTok{{-}}\NormalTok{ matrix\_B}
\end{Highlighting}
\end{Shaded}

\begin{verbatim}
matrix([[-6, -6, -6],
        [-5, -3, -1]])
\end{verbatim}

\hypertarget{transpose}{%
\subsection{Transpose}\label{transpose}}

As you progress through this book, you might see a matrix denoted as
\(A^T\); here the superscripted T stands for \emph{transpose}. If we
transpose a matrix, all we are doing is flipping the rows and columns
along the matrix's main diagonal. A visual example is much easier:

\[
\stackrel{\mbox{Matrix A}}{
\begin{bmatrix}
1_{11} & 2_{12} & 3_{13}\\
4_{21} & 5_{22} & 6_{23}
\end{bmatrix}
}
->
\stackrel{\mbox{Matrix A}^T}{
\begin{bmatrix}
1 & 4 \\
2 & 5 \\
3 & 6
\end{bmatrix}
}
\]

Like any matrix operation, a transpose is pretty easy to do when the
matrix is small; you're best bet is to rely on software to do anything
beyond a few rows or columns.

In R, all we need is the \texttt{t} function:

\begin{Shaded}
\begin{Highlighting}[]
\FunctionTok{t}\NormalTok{(matrix\_A)}
\end{Highlighting}
\end{Shaded}

\begin{verbatim}
     [,1] [,2]
[1,]    1    4
[2,]    2    5
[3,]    3    6
\end{verbatim}

In Python, we can use numpy's \texttt{transpose} method:

\begin{Shaded}
\begin{Highlighting}[]
\NormalTok{matrix\_A.transpose()}
\end{Highlighting}
\end{Shaded}

\begin{verbatim}
matrix([[1, 4],
        [2, 5],
        [3, 6]])
\end{verbatim}

\hypertarget{multiplication}{%
\subsection{Multiplication}\label{multiplication}}

Now, you probably have some confidence in doing matrix operations. Just
as quickly as we built that confidence, it will be crushed when learning
about matrix multiplication.

When dealing with matrix multiplication, we have a huge change to our
rule. No longer can our dimensions be the same! Instead, the matrices
need to be \emph{conformable} -- the first matrix needs to have the same
number of columns as the number of rows within the second matrix. In
other words, the inner dimensions must match.

Look one more time at these matrices:

\[
\stackrel{\mbox{Matrix A}}{
\begin{bmatrix}
1_{11} & 2_{12} & 3_{13}\\
4_{21} & 5_{22} & 6_{23}
\end{bmatrix}
}
.
\stackrel{\mbox{Matrix B}}{
\begin{bmatrix}
7_{11} & 8_{12} & 9_{13}\\
9_{21} & 8_{22} & 7_{23}
\end{bmatrix} 
}
\]

Matrix A has dimensions of \(2x3\), as does Matrix B. Putting those
dimensions side by side -- \(2x3 * 2x3\) -- we see that our inner
dimensions are 3 and 2 and do not match.

What if we \emph{transpose} Matrix B?

\[
\stackrel{\mbox{Matrix B}^T}{
\begin{bmatrix}
7_{11} & 9_{12} \\ 
8_{21}& 8_{22}\\
9_{31} & 7_{32}
\end{bmatrix} 
}
\]

Now we have something that works!

\[
\stackrel{\mbox{Matrix A}}{
\begin{bmatrix}
1_{11} & 2_{12} & 3_{13}\\
4_{21} & 5_{22} & 6_{23}
\end{bmatrix}
}
.
\stackrel{\mbox{Matrix B}^T}{
\begin{bmatrix}
7_{11} & 9_{12} \\ 
8_{21}& 8_{22}\\
9_{31} & 7_{32}
\end{bmatrix} 
}
=
\stackrel{\mbox{Matrix C}}{
\begin{bmatrix}
. & . \\
. & . \\
\end{bmatrix}
}
\]

Now we have a \(2x3 * 3x2\) matrix multiplication problem! The resulting
matrix will have the same dimensions as our two matrices' outer
dimensions: \(2x2\)

Here is how we will get at \(2x2\) matrix:

\[
\stackrel{\mbox{Matrix A}}{
\begin{bmatrix}
1_{11} & 2_{12} & 3_{13}\\
4_{21} & 5_{22} & 6_{23}
\end{bmatrix}
}
.
\stackrel{\mbox{Matrix B}^T}{
\begin{bmatrix}
7_{11} & 9_{12} \\ 
8_{21}& 8_{22}\\
9_{31} & 7_{32}
\end{bmatrix} 
}
=
\]

\[
\stackrel{\mbox{Matrix C}}{
\begin{bmatrix}
(A_{11}*B_{11})+(A_{12}*B_{21})+(A_{13}*B_{31}) & (A_{11}*B_{12})+(A_{12}*B_{22})+(A_{13}*B_{32}) \\
(A_{21}*B_{11})+(A_{22}*B_{21})+(A_{23}*B_{31}) & (A_{21}*B_{12})+(A_{22}*B_{22})+(A_{23}*B_{32})
\end{bmatrix} 
}
\]

That might look like a horrible mess and likely isn't easy to commit to
memory. Instead, we'd like to show you a way that might make it easier
to remember how to multiply matrices. It also gives a nice
representation of why your matrices need to be conformable.

We can leave Matrix A exactly where it is, flip Matrix B\(^T\), and
stack it right on top of Matrix A:

\[
\begin{bmatrix}
9_{b} & 8_{b} & 7_{b} \\
7_{b} & 8_{b} & 9_{b} \\
\\
1_{a} & 2_{a} & 3_{a} \\
4_{a} & 5_{a} & 6_{a}
\end{bmatrix}
\]

Now, we can let those rearranged columns from Matrix B\(^T\) ``fall
down'' through the rows of Matrix A:

\[
\begin{bmatrix}
9_{b} & 8_{b} & 7_{b} \\
\\
1_{a}*7_{b} & 2_{a}*8_{b} & 3_{a}*9_{b}\\
4_{a} & 5_{a} & 6_{a}
\end{bmatrix}
= 
\stackrel{\mbox{Matrix C}}{
\begin{bmatrix}
50 & .\\
. & .
\end{bmatrix}
}
\]

Adding those products together gives us 50 for \(C_{11}\).

Let's move that row down to the next row in the Matrix A, multiply, and
sum the products.

\[
\begin{bmatrix}
9_{b} & 8_{b} & 7_{b} \\
\\
1_{a} & 2_{a} & 3_{a}\\
4_{a}*7_{b} & 5_{a}*8_{b} & 6_{a}*9_{b}
\end{bmatrix}
= 
\stackrel{\mbox{Matrix C}}{
\begin{bmatrix}
50 & .\\
122 & .
\end{bmatrix}
}
\]

We have 122 for \(C_{21}\). That first column from Matrix B\(^T\) won't
be used any more, but now we need to move the second column through
Matrix A.

\[
\begin{bmatrix}
1_{a}*9_{b} & 2_{a}*8_{b} & 3_{a}*7_{b}\\
4_{a} & 5_{a} & 6_{a}
\end{bmatrix}
= 
\stackrel{\mbox{Matrix C}}{
\begin{bmatrix}
50 & 46\\
122 & .
\end{bmatrix}
}
\]

That gives us 46 for \(C_{12}\).

And finally:

\[
\begin{bmatrix}
1_{a} & 2_{a} & 3_{a}\\
4_{a}*9_{b} & 5_{a}*8_{b} & 6_{a}*7_{b}
\end{bmatrix}
=
\stackrel{\mbox{Matrix C}}{
\begin{bmatrix}
50 & 46\\
122 & 118
\end{bmatrix}
}
\]

We have 118 for \(C_{22}\).

Now that you know how these work, you can see how easy it is to handle
these tasks in R and Python.

In R, we need to use a fancy operator: \texttt{\%*\%}. This is just R's
matrix multiplication operator. We will also use the transpose function:
\texttt{t}.

\begin{Shaded}
\begin{Highlighting}[]
\NormalTok{matrix\_A }\SpecialCharTok{\%*\%} \FunctionTok{t}\NormalTok{(matrix\_B)}
\end{Highlighting}
\end{Shaded}

\begin{verbatim}
     [,1] [,2]
[1,]   50   46
[2,]  122  118
\end{verbatim}

In Python, we can just use the regular multiplication operator and the
transpose method:

\begin{Shaded}
\begin{Highlighting}[]
\NormalTok{matrix\_A }\OperatorTok{*}\NormalTok{ matrix\_B.transpose()}
\end{Highlighting}
\end{Shaded}

\begin{verbatim}
matrix([[ 50,  46],
        [122, 118]])
\end{verbatim}

You can see that whether we do this by hand, R, or Python, we come up
with the same answer! While these small matrices can definitely be done
by hand, we will always trust the computer to handle larger matrices.

\hypertarget{inversion}{%
\subsection{Inversion}\label{inversion}}

You might want to think of \emph{matrix inversion} as the reciprocal of
the matrix, usually noted as \(A^{-1}\). The biggest reason that we
might invert a matrix is because there is no matrix division.

Inversion can only be performed on square matrices (e.g., \(2x2\),
\(3x3\), \(4x4\)) and the \emph{determinant} of a matrix cannot be 0.
Since the determinant is important for finding the inverse, we should
probably have an idea about how to find the determinant.

\hypertarget{matrix-determinant}{%
\subsubsection{Matrix Determinant}\label{matrix-determinant}}

While we've been using the matrix row/column positions in our examples,
we are going to shift to letters to label the positions. We can start
with a \(2x2\) matrix:

\[
\stackrel{\mbox{Matrix C}}{
\begin{bmatrix}
A & B\\
C & D
\end{bmatrix}
}
\]

To find the determinant, we would take \(\mid C \mid = (A*D) - (B*C)\).

Returning back to Matrix C, we have
\(\mid C \mid = (50_a*118_d) - (46_b*122_c) = 288\)

\[
\stackrel{\mbox{Matrix C}}{
\begin{bmatrix}
50 & 46\\
122 & 118
\end{bmatrix}
}
\]

A \(3x3\) matrix doesn't pose much more of a challenge.

\[
\stackrel{\mbox{Matrix D}}{
\begin{bmatrix}
A & B & C\\
D & E & F\\
G & H & I
\end{bmatrix}
}
\]

The canonical form might not be as intuitive, but it is worth seeing:

\[
\mid D \mid = A\begin{vmatrix}
E & I\\
F & H
\end{vmatrix}  - 
B\begin{vmatrix}
D & I\\
F & G
\end{vmatrix} + 
C\begin{vmatrix}
D & H\\
E & G
\end{vmatrix}
\]

Breaking it down a bit further will help to see where all of the values
go:

\[
\mid D \mid = A(E*I - F*H) - B(D*I - F*G) + C(D*H - E*G)
\]

Now we can work that out with a real matrix:

\[
\stackrel{\mbox{Matrix D}}{
\begin{bmatrix}
2 & 1 & 3\\
6 & 5 & 4\\
7 & 8 & 9
\end{bmatrix}
}
\]

To get our determinant:

\[
\mid D \mid = 2(5*9 - 4*8) - 1(6*9 - 4*7) + 3(6*8 - 5*7) = 39
\]

And just to confirm that our math is correct, we can check for the
determinant in R and Python.

R has a handy function called \texttt{det}:

\begin{Shaded}
\begin{Highlighting}[]
\NormalTok{matrix\_D }\OtherTok{\textless{}{-}} \FunctionTok{matrix}\NormalTok{(}\FunctionTok{c}\NormalTok{(}\DecValTok{2}\NormalTok{, }\DecValTok{1}\NormalTok{, }\DecValTok{3}\NormalTok{,}
                     \DecValTok{6}\NormalTok{, }\DecValTok{5}\NormalTok{, }\DecValTok{4}\NormalTok{,}
                     \DecValTok{7}\NormalTok{, }\DecValTok{8}\NormalTok{, }\DecValTok{9}\NormalTok{), }
                   \AttributeTok{nrow =} \DecValTok{3}\NormalTok{, }
                   \AttributeTok{ncol =} \DecValTok{3}\NormalTok{, }
                   \AttributeTok{byrow =} \ConstantTok{TRUE}\NormalTok{)}

\FunctionTok{det}\NormalTok{(matrix\_D)}
\end{Highlighting}
\end{Shaded}

\begin{verbatim}
[1] 39
\end{verbatim}

We can keep using \texttt{numpy}, but we will have to use \texttt{det}
within the \texttt{linalg} module.

\begin{Shaded}
\begin{Highlighting}[]
\NormalTok{matrix\_D }\OperatorTok{=}\NormalTok{ np.matrix(}\StringTok{\textquotesingle{}2 1 3; 6 5 4; 7 8 9\textquotesingle{}}\NormalTok{)}

\NormalTok{np.linalg.det(matrix\_D)}
\end{Highlighting}
\end{Shaded}

\begin{verbatim}
38.99999999999999
\end{verbatim}

Just to show you how this pattern would continue

You can find a lot of examples online on how to do \(2x2\) and \(3x3\)
matrix inversions, mostly because they are the easiest to do.

How do you know that you properly inverted your matrix? You multiply the
original matrix by the inverse matrix and you will get an
\emph{identity} matrix.

We have a nice figure in Figure @ref(fig:hello), and also a table in
Table @ref(tab:iris).



\end{document}
